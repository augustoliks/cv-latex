%%%%%%%%%%%%%%%%%
% This is an sample CV template created using altacv.cls
% (v1.1.4, 27 July 2018) written by LianTze Lim (liantze@gmail.com). Now compiles with pdfLaTeX, XeLaTeX and LuaLaTeX.
% 
%% It may be distributed and/or modified under the
%% conditions of the LaTeX Project Public License, either version 1.3
%% of this license or (at your option) any later version.
%% The latest version of this license is in
%%    http://www.latex-project.org/lppl.txt
%% and version 1.3 or later is part of all distributions of LaTeX
%% version 2003/12/01 or later.
%%%%%%%%%%%%%%%%
\newcommand{\RNum}[1]{\uppercase\expandafter{\romannumeral #1\relax}}
%% If you need to pass whatever options to xcolor
\PassOptionsToPackage{dvipsnames}{xcolor}

%% If you are using \orcid or academicons
%% icons, make sure you have the academicons 
%% option here, and compile with XeLaTeX
%% or LuaLaTeX.
% \documentclass[10pt,a4paper,academicons]{altacv}

%% Use the "normalphoto" option if you want a normal photo instead of cropped to a circle
% \documentclass[10pt,a4paper,normalphoto]{altacv}

\documentclass[10pt,a4paper]{altacv}
%% AltaCV uses the fontawesome and academicon fonts
%% and packages. 
%% See texdoc.net/pkg/fontawecome and http://texdoc.net/pkg/academicons for full list of symbols.
%% 
%% Compile with LuaLaTeX for best results. If you
%% want to use XeLaTeX, you may need to install
%% Academicons.ttf in your operating system's font 
%% folder.


% Change the page layout if you need to
\geometry{left=1cm,right=9cm,marginparwidth=6.8cm,marginparsep=1.2cm,top=1.25cm,bottom=1.25cm,footskip=2\baselineskip}

% Change the font if you want to.

% If using pdflatex:
\usepackage[T1]{fontenc}
\usepackage[utf8]{inputenc}
\usepackage[default]{lato}

% If using xelatex or lualatex:
% \setmainfont{Lato}

% Change the colours if you want to
\definecolor{Navy}{HTML}{000080}
\definecolor{SlateGrey}{HTML}{2E2E2E}
\definecolor{LightGrey}{HTML}{444444}
\colorlet{heading}{Navy}
\colorlet{accent}{Navy}
\colorlet{emphasis}{SlateGrey}
\colorlet{body}{LightGrey}

% Change the bullets for itemize and rating marker
% for \cvskill if you want to
\renewcommand{\itemmarker}{{\small\textbullet}}
\renewcommand{\ratingmarker}{\faCircle}
%% sample.bib contains your publications
\addbibresource{sample.bib}

\usepackage[colorlinks]{hyperref}

\begin{document}

\name{Carlos Augusto dos Santos Neto}
\tagline{Objetivo: Desenvolvedor Backend - Júnior}
\personalinfo{
    \email{\href{mailto:carlos.neto.dev@gmail.com}{carlos.neto.dev@gmail.com}}
    \phone{\href{tel:5512987078145}{+55 12 987078145}}
    \linkedin{\href{https://www.linkedin.com/in/carlos-neto-15494213b/}{Carlos Neto}}
    \github{\href{https://github.com/augustoliks}{augustoliks} }
    \location{\href{https://maps.google.com/maps?q=-23.296193,-46.027498}{Jacareí-SP}}
}


%% Make the header extend all the way to the right, if you want. 
\begin{fullwidth}
\makecvheader
\end{fullwidth}

%% Depending on your tastes, you may want to make fonts of itemize environments slightly smaller
% \AtBeginEnvironment{itemize}{\small}


%% Provide the file name containing the sidebar contents as an optional parameter to \cvsection.
%% You can always just use \marginpar{...} if you do
%% not need to align the top of the contents to any
%% \cvsection title in the "main" bar.
\cvsection[page1sidebar]{Experiência}

\cvevent{Programador Júnior 2}{Fotosensores Tecnologias LTDA}{Agosto 2020 -- Atual}{}
\begin{itemize}
    \item Desenvolvimento e testes com a linguagem Python;
    \item Elaboração de Arquitetura de Sistemas;
    \item Design e implementação de APIs;
    \item Rotinas de entrada, filtro/transformação e saída de logs;
    \item Empacotamento de bibliotecas e Softwares;
    \item Otimização de imagens de containers;
    \item Automação e proviosionamento de sistemas;
    \item Criação e padronização de rotinas CI/CD;
    \item Soluções de DevOps e colaborações de IaC;
    \item Elaboração de Provas de Conceitos;
    \item Padronização de documentações.
\end{itemize}

\divider

\cvevent{Programador Júnior 1}{Fotosensores Tecnologias LTDA}{Junho 2019 -- Agosto 2020}{}
\begin{itemize}
    \item Desenvolvimento e testes com a linguagem Python;
    \item Integrações entre sistemas;
    \item Criação, coleta e exibição de métricas de Sistemas e  Softwares;
    \item Monitoramento de Softwares e Sistemas;
    \item Configuração de Proxy Reverso;
    \item Documentação de procedimentos técnicos e operacionais.
\end{itemize}

\cvsection{Projetos}

\cvevent{Mapbix: Sistema de Monitoramento de Equipamentos Fiscalizadores de Trânsito}
{\MakeLowercase{python, flask, vue.js, zabbix, grafana, nginx}}{}{}
\begin{itemize}
    \item Trabalho de Graduação na Fatec São José dos Campos;
    \item Atua como produto da empresa Fotosensores Tecnologia LTDA.
\end{itemize}
\divider

\cvevent{PoC: realtime-log-web-viewer}
{\MakeLowercase{python, fastAPI, rsyslog, docker, redis}}{}{}
\begin{itemize}
    \item PoC de exibição de logs de aplicações em tempo real em um ambiente Web;
    \item \github{\href{https://github.com/augustoliks/realtime-log-web-viewer}{https://github.com/augustoliks/realtime-log-web-viewer}}
\end{itemize}
\divider

\cvevent{Biblioteca Python: gelfguru}
{\MakeLowercase{python, Travis-CI, poetry}}{}{}
\begin{itemize}
    \item Adaptor de logs de aplicaçãoes Python para formato GELF;
    \item \href{https://pypi.org/project/gelfguru/}{https://pypi.org/project/gelfguru/}
    \item \github{\href{https://github.com/augustoliks/loguru-gelf-extension}{https://github.com/augustoliks/loguru-gelf-extension}}
\end{itemize}


\clearpage

%% If the NEXT page doesn't start with a \cvsection but you'd
%% still like to add a sidebar, then use this command on THIS
%% page to add it. The optional argument lets you pull up the 
%% sidebar a bit so that it looks aligned with the top of the
%% main column.
% \addnextpagesidebar[-1ex]{page3sidebar}

\end{document}
